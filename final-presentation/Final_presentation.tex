\documentclass[t]{beamer}

\usepackage[T2A]{fontenc}
\usepackage[cp1251]{inputenc}
\usepackage[english]{babel}
\usepackage{amssymb,amsmath,amsthm,amsfonts}
\usepackage{graphicx}
\defbeamertemplate*{footline}{Warsaw} {%
\leavevmode%
\hbox{%
\begin{beamercolorbox}[wd=.58\paperwidth,ht=2.5ex,dp=1.125ex,leftskip=.3cm,rightskip=.3cm]{author in head/foot}%
\insertframenumber{}%
\hfill\insertshortauthor{}
\end{beamercolorbox}%
%%\begin{beamercolorbox}[wd=.42\paperwidth,ht=2.5ex,dp=1.125ex,leftskip=.2cm,rightskip=.2cm]{title in head/foot}%
%%\usebeamerfont{title in head/foot}
%%\insertshorttitle{Efficient First Order Inductive Learner on Spark}
%%\end{beamercolorbox}
}%
\vskip0pt%
}
\title{\qquad Distributed Big Data Analytics\qquad\qquad\qquad\\ Efficient First Order Inductive Learner on Spark}
\author{ Georgiy Shurkhovetskyy, Eskender Haziiev \\ Supervisors: Hajira Jabeen, Gezim Sejdiu}
\institute{University of Bonn, 2017}

\date{}



\begin{document}

%------------------------------------------------ 1
\begin{frame}
%\transdissolve[duration=0.4]
  \titlepage
\end{frame}







%%------------------------------------------------------------------- 2
\begin{frame}{Problem Statement}
\footnotesize

\textbf{Definition 1}. A literal $L$ may be predicate $P$ or
 $\neg{P}$.
\pause

\textbf{Definition 2}. A clause body is a conjunction of literals.
\pause

\textbf{Definition 3}. A clause head is predicate.
\pause

\textbf{Definition 4}. A Horn clause consists of a head and a
 body:
\begin{equation}\label{eq0}
P\leftarrow L_1,L_2,..., L_n .
\end{equation}
\textbf{Definition 5}. A learning rule for predicate $P$ is
 a collection of Horn clauses each with head $P$.

The predicates can be defined \emph{extensionally} as a list of
tuples for which the predicate is true, or \emph{intensionally} as
a set of Horn clauses.
\pause

\textbf{Definition 6}. $k$-tuple $<a_1, a_2, \ldots, a_k>$ is a
finite sequence of $k$ constants.
\pause

\textbf{Definition 7}. A tuple satisfies a learning rule if it
satisfies one of the Horn clauses of this rule.

For every relation a set of the $\oplus$tuples  which
 belong to this relation is given.

For a target relation a set of $\oplus$tuples is also given. A set
of the $\ominus$tuples not belonging to the target relation may be
given. The statement is introduced: if some tuple is not included
in set $\oplus$tuples then it is $\ominus$tuple.
\pause

\textcolor[rgb]{0.00,0.00,1.00}{\textbf{Problem statement:}} Using
Spark framework realize FOIL to find a learning rule as a set of
Horn clauses for the target relation that consistent with given
positive examples and not cover any given negative examples.
\end{frame}

%%------------------------------------------------------------------------------------------ 3
\begin{frame}{Approach}
Foil algorithm is realized using Scala in Spark framework.
\begin{figure}[h!]\centering
\includegraphics[scale =0.40]{approach.png}
\end{figure}

\pause
\vspace{-8pt} \footnotesize{To select literal with greatest
positive gain:}
\begin{equation}\label{eq4}
gain (L_{i})=n_i^{\oplus\oplus} \cdot (I(T_i)-I(T_{i+1})),\quad
I(T_i)=-\log_2 (n_i^\oplus/(n_i^\oplus+n_i^\ominus)),
\end{equation}
$n_i^\oplus$ --- a number of $\oplus$tuples in $T_i$,
$n_i^\ominus$ --- a number of $\ominus$tuples in $T_i$,\\
$\!\!\!\!n_i^{\oplus\oplus}$ --- a number of the $\oplus$tuples in
$T_i$ represented by one or more tuples in $T_{i+1}$.
\end{frame}


%%------------------------------------------------------------------------------------------ 4
\begin{frame}{Implementation}
\begin{figure}[t]\centering
\includegraphics[scale =0.39]{shema.png}
\end{figure}
\end{frame}

%%------------------------------------------------------------------------------------------ 5
\begin{frame}{Evaluation}
\vspace{-2pt}
\textcolor[rgb]{0.00,0.00,1.00}{\underline{Evaluation metrics:}}  $\rho_k(A_1,A_2)=|n_k(A_1)-n_k(A_2)|$, $k=1,2,3$.

\footnotesize{Indicators:}
$n_1(A)=\frac{n^\oplus_{\textrm{covered}}(A)}{n^\oplus_{\textrm{given}}}$, $n_2(A)=\frac{n^\ominus_{\textrm{covered}}(A)}{n^\ominus_{\textrm{given}}}$, $n_3(A)=\textrm{runtime}(A)$.
\pause

\vspace{2pt}
\textcolor[rgb]{0.00,0.00,1.00}{Experiment 1.} (a), \underline{results:} $daughter(X_1,X_2) \leftarrow female(X_1) parent(X_2,X_1)$;\\ covered $\oplus$ examples: (Emily, Tom), (Mary, Ann); covered $\ominus$ examples: $\emptyset$.
\pause

\textcolor[rgb]{0.00,0.00,1.00}{Experiment 2.} (b), \underline{results:} $daughter(X_1,X_2) \leftarrow female(X_1) parent(X_2,X_1)$;\\ covered $\oplus$ examples: (Opra, Ann), (Kate, Ann), (Lina, Kate), (Gail, Hovard); covered $\ominus$ examples: $\emptyset$.
\pause

\textcolor[rgb]{0.00,0.00,1.00}{Experiment 3.} (b), \underline{results:} $son(X_1,X_2) \leftarrow male(X_1) parent(X_2,X_1)$;\\ covered $\oplus$ examples: (Ryan, Uve), (Bill, Opra), (Hovard, Ann), (Nick, Kate), (Uve, Ann), (Tom, Uve); covered $\ominus$ examples: $\emptyset$.
\vspace{-8pt}
\begin{figure}[h!] \centering
\includegraphics[scale =0.28]{tree_ds.png}
\end{figure}
\vspace{-10pt}
\begin{figure}[h!] \centering
\includegraphics[scale =0.3]{Table2.png}
\end{figure}
\end{frame}

%%------------------------------------------------------------------------------------------ 6
\begin{frame}{Conclusion and future work}
\footnotesize \textcolor[rgb]{0.00,0.00,1.00}{Results of presented
project are:}
\pause

1. FOIL algorithm is implemented in Spark:

\qquad (i) the developed application gets the input data in
extensionally form (i.e. consists of names and $\oplus$ and
$\ominus$ tuples for target predicate, and names and $\oplus$
tuples for background predicates;)
\pause

\qquad (ii) the results are obtained  both in intensional form
(i.e. as Horn clause for the target relation that is consistent with
given $\oplus$tuples and does not cover any given $\ominus$tuples) and
 extensional form.
\pause

2. To evaluate developed application the metrics based on three
efficiency indicators are introduced.
\pause

3. Three experiments show implementation performs at sufficient efficiency.
\pause


\vspace {12pt}


\textcolor[rgb]{0.00,0.00,1.00}{The future steps might include:}
\pause

1. Carrying out experiments with a more complex data structure,
for example, when a tree node is described by a large number of
predicates;
\pause

2. Evaluation of the application efficiency compared to other
implementations of the Foil algorithm.
\end{frame}

%%------------------------------------------------------------------------------------------ 7
\begin{frame}{Lessons Learned}

Besides the detailed study of the FOIL algorithm

we also consider as the results of this project the following:
\pause
\vspace {12pt}

1. The extension of our general skills on applications development using
the Spark framework and the Scala programming language;
\pause

2. Gaining experience in project testing and selection of evaluation datasets.
\pause

3. Getting deeper understanding of how and when to apply distributed means of data processing.

\end{frame}

%%------------------------------------------------------------------------------------------ 7
\begin{frame}{References}
\textcolor[rgb]{0.00,0.00,1.00}{[1]} T. Horvath. Learning in logic V. In Learning from Non-Standard Data, WS 2016/17.
University of Bonn, Fraunhofer IAIS, Sankt Augustin, Germany, 2016.

\textcolor[rgb]{0.00,0.00,1.00}{[2]} S. Muggleton and L. De Raedt. Inductive logic programming: Theory and methods.
The Journal of Logic Programming, 19(20):629�679, 1994.

\textcolor[rgb]{0.00,0.00,1.00}{[3]} M. Pazzani and D. Kibler. The utility of knowledge in inductive learning. Machine
Learning, 9(1):57�94, 1992.

\textcolor[rgb]{0.00,0.00,1.00}{[4]} J.R. Quinlan. Learning logical definitions from relations. Machine Learning, (5):239�266,
1990.

\textcolor[rgb]{0.00,0.00,1.00}{[5]} J.R. Quinlan and R.M. Cameron-Jones. Foil: A midterm report. In Machine Learning:
ECML-93, pages 3�20. Springer-Verlag, Berlin, Heidelberg, 1993.

\textcolor[rgb]{0.00,0.00,1.00}{[6]} J.R. Quinlan and R.M. Cameron-Jones. Induction of logic programs: Foil and related
systems. New Generation Computing, 13(3):287�312, 1995.
\end{frame}

\end{document}
